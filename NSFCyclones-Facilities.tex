\documentclass[11pt]{article}
%\documentclass{ametsoc}
\usepackage{natbib}
\usepackage[margin=1.00in]{geometry}
\usepackage{graphicx}
\usepackage{wrapfig}
\usepackage{rotating}
\usepackage{amsmath}
\usepackage{graphicx}
\usepackage{multicol}
\usepackage{natbib}
\usepackage{wrapfig}
\usepackage{hyperref}
\usepackage{tabularx}
\usepackage{setspace}
\usepackage{comment}
\usepackage{bibentry}

\begin{document}

\setcounter{section}{2}

\appendix

\setcounter{section}{8}

\section{Facilities, Equipment and Resources}

In general, all of the offices used by members of the proposed research team will have computers stations with access to internet, as well as phones, printers and copiers. Both Stony Brook University and UC Davis provide IT and administrative support.

\ \\

\noindent Stony Brook University: PI Reed has an academic office with full computer access, internet connection, phone and IT support in the School of Marine and Atmospheric Sciences at Stony Brook University. Prof. Reed also has access to video conferencing to provide updates and direction to the funded collaborators at UC Davis. In addition, the Graduate Student Researcher will be provided adequate workspace complete with a desk, telephone and internet in the School of Marine and Atmospheric Sciences. 

\ \\

\noindent UC Davis: Co-PI Ullrich has an office space provided at UC Davis, including computer, monitors, high-speed internet and access to printers and copiers. In addition, the UC Davis group will have access to video conferencing in order to communicate with the team at Stony Brook University to discuss research progress.  The UC Davis Graduate Student Researcher will also be provided a desk, telephone, internet, etc.

\ \\

\noindent No equipment is needed for the proposed research, besides computer environments for both the Graduate Student Researchers. Furthermore, access to supercomputers for the proposed CAM5-SE variable-resolution ensemble simulations are expected to be supplied by NSF and/or National Center for Atmospheric Research through a separate computational resources proposal, pending approval of final computational resources needed.

\ \\

\noindent As part of the proposed research the teams at Stony Brook University and UC Davis will work closely with external collaborators.  In particular, Dr. Michael Wehner at the Department of Energy Lawrence Berkeley National Laboratory will provide the CAM5-FV AMIP and CLIVAR model data and aid in the analysis of TCs for the proposed work.  Dr. Wehner has considerable expertise in running 25 km simulations with numerous versions of CAM. Furthermore, Dr. Julio Bacmeister at the NSF supported National Center for Atmospheric Research is heavily involved with the development of the high-resolution versions of CAM5-SE and will provide access to the CAM5 AMIP and RCP 8.5 simulations that were performed under his direction.  In addition, Dr. Bacmeister will aid in the set up of the proposed CAM5-SE ensemble simulations

\end{document}