\documentclass[11pt]{article}

\usepackage{amsmath}
\usepackage{graphicx}
\usepackage{multicol}
\usepackage{natbib}
\usepackage{wrapfig}
\usepackage{hyperref}
\usepackage{tabularx}
\usepackage{setspace}
\usepackage{comment}
\usepackage{sectsty}

\oddsidemargin 0cm
\evensidemargin 0cm

\usepackage[compact]{titlesec}  
\titlespacing{\section}{0pt}{0pt}{-5pt}
\titlespacing{\section}{0pt}{0pt}{-5pt}
\titlespacing{\subsection}{0pt}{0pt}{-5pt}

\sectionfont{\fontsize{12}{15}\selectfont}
\subsectionfont{\fontsize{11}{15}\selectfont}

\usepackage[margin=1in]{geometry}

\parindent 0cm
\parskip 0.5cm

\usepackage{fancyhdr}
\pagestyle{plain}
%\fancyhf{}
%\fancyhead[L]{AOSS Reference Sheet}
%\fancyhead[CH]{test}
\fancyfoot[C]{Page \thepage}


\begin{document}

{\Large \textbf{Budget Justification}}

\section{Personnel: \$118,970}
Salaries are based on Fiscal Year 2014/2015 levels and assume a 3\% increase in each subsequent fiscal year.

\subsection{Salaries: \$113,314}

\subsubsection{Principal Investigator (Dr. Paul Ullrich): \$24,260}

PI Dr. Paul Ullrich will be direct charging the project 1 month (8.33\%) of additional compensation during the summer months in project years 1, 2 and 3. Anticipating a forthcoming merit increase, this amounts to salary costs of \$7,849 in year 1, \$8,084 in year 2 and \$8,327 in year 3 or 1/11 of his annual salary each year based on 11/12 appointment.

\subsubsection{Graduate Student Researcher IV (Meina Wang): \$89,054}

One graduate student researcher (GSR IVs; 9 months at 48\%, 3 summer months at 100\%) will be funded for the duration of this project, conducted as part of their degree with a thesis option. The current annual salary rate for a GSR IV is \$45,300.  Based on 9 months at 48\% and 3 months at 100\% this amounts to a total of \$28,811 in project year 1, \$29,676 in year 2 and \$30,567 in year 3.

\subsection{Benefits: \$5,656}
 
Employee Benefits are based on Federally Approved Composite Benefit Rates. The University of California's current Composite Benefit Rates have been federally reviewed and approved through June 30, 2015.  Rates for Fiscal Years 2015/2016 and 2016/2017 are projected while the rates for Fiscal Year 2017/2018 and beyond are estimates using a 3\% increase over the preceding year�s rates as the basis of the estimate. 

\subsubsection{Principal Investigator (Dr. Ullrich): \$4,499}
 
As the direct charging of salary for Dr. Ullrich is during the summer of each project year, the projected composite benefit rates are 17\% and 18\% for project years 1 and 2 respectively and an estimated rate of 18.5\% in year 3.
 
\subsubsection{Graduate Student Researcher IV: \$1,157}
The composite benefit rate for the GSR is expected to remain at 1.3\% for the duration of the project.

\section{Travel: \$27,000}
The budget includes domestic travel costs associated with one annual trip to the American Geophysical Union (AGU) Fall Meeting in San Francisco, CA for the PI and GSR. Including registration fees, travel expenses and meal and incidental expenses, the total cost is estimated at \$1,400 per traveler per year (5 days, \$60 travel, \$400 registration, \$700 hotel, \$240 meals and incidentals). Domestic travel expenses are also included for the GSR to present at the American Meteorological Society (AMS) each year, estimated at \$1,800 per year (4 days, \$300 airfare, \$400 registration, \$800 hotel, \$300 meals and incidentals). International travel expenses are included to the European Geophysical Union (EGU) in year 1 and 3 and the International Association of Meteorology and Atmospheric Sciences (IAMAS) conference in Cape Town, South Africa in year 2, estimated at \$4,400 for one participant (5 days, \$2000 airfare, \$400 registration, \$1200 hotel, \$800 meals and incidentals).

\section{Supplies: \$4,818}
A one-time expense of \$3,000 will cover the cost of a laptop for use by the GSR over the duration of the project.  The GSR will require this computing environment for performing work away from campus.  Additionally the project requires the purchase of software licenses totaling \$1,818.  The software license charges are \$138 / year / license for the mathematics software package Maple and \$165 / year / license for the software package Matlab, for a total of \$303 / year / license. Software will be provided to the GSR (1 license) and PI (1 license) over the duration of the project. This amounts to \$606 / year for all licenses. These software packages will be used by the students and PI for data processing, analysis and modeling.

\section{Other Expenses: \$66,224}

\subsection{Publication Charges: \$6,000}

Publication costs are incurred from publication of work produced by this project and the associated annual cost is expected to be \$2,000 per year (16 pages at \$125 per page).

\subsection{Tuition and Fee Costs: \$60,224}

The GSR IV is expected to have residency status which amounts to \$18,195 in year 1, \$20,014 in year 2 and \$22,016 in year 3.

\section{Indirect Cost: \$89,171}

Per the University of California, Davis� Federally approved Indirect Cost Rate for on-campus research, an initial rate of 56.5\% Modified Total Direct Cost will be applied to the project during the period through June 30, 2016 and 57\% for the period of July 1, 2016 through June 30, 2017.  As overhead rates have not been established beyond Fiscal Year 2016/2017 the last approved rate of 57\% is used for the remainder of the project for budgeting purposes.  The Negotiated Indirect Cost Rate Agreement can be found here:  \url{http://research.ucdavis.edu/wp-content/uploads/F-A-Rate-Agreement-2014-update.pdf}.

\end{document}
