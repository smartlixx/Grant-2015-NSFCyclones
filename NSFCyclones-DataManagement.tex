\documentclass[11pt]{article}
%\documentclass{ametsoc}
\usepackage{natbib}
\usepackage[margin=1.00in]{geometry}
\usepackage{graphicx}
\usepackage{wrapfig}
\usepackage{rotating}
\usepackage{amsmath}
\usepackage{graphicx}
\usepackage{multicol}
\usepackage{natbib}
\usepackage{wrapfig}
\usepackage{hyperref}
\usepackage{tabularx}
\usepackage{setspace}
\usepackage{comment}
\usepackage{bibentry}

\begin{document}

\appendix

\setcounter{section}{9}

\section{Data Management and Communication}

Data from the already completed runs by NCAR (CAM5-SE) and LBNL (CAM5-FV) have already been made available to the research team and will continue to be managed by the individual groups that performed the simulations. Data from the ensemble variable-resolution CAM5-SE runs, 10 in total, proposed by the project will come from a separate NSF Yellowstone (or other available supercomputers) allocation proposal for the required core-hours. The output from these runs will be archived on NCAR HPSS, where it will be available for community use. If during the proposed research certain simulations are found to be useful to the broader community, the model data for the specified runs can be published to the Earth System Grid for broad dissemination. This archive allows for public access to model output from all simulations and consequently allows for these results to be leveraged for future study.  The data will be post-processed and made available in different temporal resolutions (i.e. daily, monthly, seasonal, and climate) using the Climate Data Operators (CDO) and NetCDF Operators (NCO).

\ \\

\noindent All public data will also be deposited in Merritt, a repository service from the University of California Curation Center (UC3) that has capabilities to manage, archive and share digital content. Merritt allows access to the public via persistent URLs, provides tools for long-term data management, and permits permanent storage options. Merritt has built-in contingencies for disaster recovery including redundancy and recovery plans.

\bibliographystyle{wileyqj}  
\nobibliography{NSFCyclones-Bibliography}

\end{document} 