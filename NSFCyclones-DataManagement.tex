\documentclass[11pt]{article}
%\documentclass{ametsoc}
\usepackage{natbib}
\usepackage[margin=1.00in]{geometry}
\usepackage{graphicx}
\usepackage{wrapfig}
\usepackage{rotating}
\usepackage{amsmath}
\usepackage{graphicx}
\usepackage{multicol}
\usepackage{natbib}
\usepackage{wrapfig}
\usepackage{hyperref}
\usepackage{tabularx}
\usepackage{setspace}
\usepackage{comment}
\usepackage{bibentry}

\begin{document}

\appendix

\setcounter{section}{9}

\section{Data Management and Communication}

Data from the already completed runs by NCAR (CAM5-SE) and LBNL (CAM5-FV) have already been made available to the research team and will continue to be managed by the individual groups that performed the simulations. Data from the ensemble variable-resolution CAM5-SE runs, 10 in total, proposed by the project will come from a separate NSF Yellowstone (or other available supercomputers) allocation proposal for the required core-hours. The output from these runs will be archived on NCAR HPSS. Furthermore, we anticipate releasing all variable resolution model runs completed by this project via the Earth System Grid (\texttt{https://www.earthsystemgrid.org/}). This archive allows for public access to model output from all simulations and consequently allows for these results to be leveraged for future study. The data will be post-processed and made available in different temporal resolutions (i.e. daily, monthly, seasonal, and climate) using the Climate Data Operators (CDO) and NetCDF Operators (NCO).

\ \\

\noindent All software produced in conjunction with the project (including the AEW, TC and ETC detection capability in TempestExtremes) will be released under the open source Lesser GNU Public License (LGPL), so as to be available for use by other investigators. The software will be available for public download via GitHub. We further plan to release a summary of all results from this work for public consumption.

\bibliographystyle{wileyqj}  
\nobibliography{NSFCyclones-Bibliography}

\end{document} 