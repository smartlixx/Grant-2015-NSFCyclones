\documentclass[11pt]{article}
%\documentclass{ametsoc}
\usepackage{natbib}
\usepackage[margin=1.00in]{geometry}
\usepackage{graphicx}
\usepackage{wrapfig}
\usepackage{rotating}
\usepackage{amsmath}
\usepackage{graphicx}
\usepackage{multicol}
\usepackage{natbib}
\usepackage{wrapfig}
\usepackage{hyperref}
\usepackage{tabularx}
\usepackage{setspace}
\usepackage{comment}
\usepackage{bibentry}

\begin{document}

\appendix

\setcounter{section}{9}

\section{Data Management and Communication}

Data from both the WRF runs and VRGCM runs will be post-processed and made available for future use via the Earth System Grid \citep{williams2009earth}. This archive allows for public access to model output from all simulations and consequently allows for these results to be leveraged for future study.  The data will be post-processed and made available in different temporal resolutions (i.e. daily, monthly, seasonal, and climate) using the Climate Data Operators (CDO) and NetCDF Operators (NCO) \citep{schulzweida2007cdo,zender2008analysis}.

\ \\

\noindent All public data will also be deposited in Merritt, a repository service from the University of California Curation Center (UC3) that has capabilities to manage, archive and share digital content. Merritt allows access to the public via persistent URLs, provides tools for long-term data management, and permits permanent storage options. Merritt has built-in contingencies for disaster recovery including redundancy and recovery plans.

\bibliographystyle{wileyqj}  
\nobibliography{NSFCyclones-Bibliography}

\end{document} 