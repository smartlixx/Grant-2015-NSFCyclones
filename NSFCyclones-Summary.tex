\documentclass[11pt]{article}

\usepackage{amsmath}
\usepackage{graphicx}
\usepackage{multicol}
\usepackage{natbib}
\usepackage{wrapfig}
\usepackage{hyperref}
\usepackage{tabularx}
\usepackage{setspace}

\oddsidemargin 0cm
\evensidemargin 0cm

\usepackage[margin=1in]{geometry}

\parindent 0cm
\parskip 0.5cm

\usepackage{fancyhdr}
\pagestyle{plain}
%\fancyhf{}
%\fancyhead[L]{AOSS Reference Sheet}
%\fancyhead[CH]{test}
\fancyfoot[C]{Page \thepage}

\newcommand{\vb}{\mathbf}
\newcommand{\diff}[2]{\frac{d #1}{d #2}}
\newcommand{\diffsq}[2]{\frac{d^2 #1}{{d #2}^2}}
\newcommand{\pdiff}[2]{\frac{\partial #1}{\partial #2}}
\newcommand{\pdiffsq}[2]{\frac{\partial^2 #1}{{\partial #2}^2}}
\newcommand{\topic}{\textbf}
\newcommand{\arcsinh}{\mathrm{arcsinh}}
\newcommand{\arccosh}{\mathrm{arccosh}}
\newcommand{\arctanh}{\mathrm{arctanh}}

\begin{document}

\appendix

\addtocounter{section}{1}

\section{Project Summary}
\vspace{-0.2cm}

%\vspace{-0.7cm}
\subsection*{Overview}
\vspace{-0.6cm}

The next century will see unprecedented changes to the climate system with direct socioeconomic repercussions.  These impacts will be particularly apparent from extreme weather events: As stated in the Third National Climate Assessment, ``changes in extreme weather events are the primary way that most people experience climate change.''  In this work, we propose to investigate the complete lifecycle of one particular type of extreme weather event -- \textbf{North Atlantic basin tropical cyclones}.  Storms in this region are among the most expensive natural disasters in the United states, with economic and social costs from flooding, property damage and fatality.  Our goal is to understand the characteristics of these storms under projections of future climate from cyclogenesis, through mid-life and termination (either via landfall or extratropical transition).  This proposal further aims to explain the physical drivers behind reported reductions in North Atlantic tropical cyclone counts that have been observed in past studies.

The overarching objectives of this proposal are as follows:  First, the development of robust methods for detection and characterization of African wasterly waves, tropical cyclones and extratropical cyclones within a single framework, and provide these tools for analysis of large climate datasets.  Second, the development of a complete catalogue of past and projected cyclones in the North Atlantic basin, including their character from genesis through extratropical transition or landfall.  Third, a scientifically and statistically robust risk assessment of the effect of projected human-induced climate change on North Atlantic cyclones.

\vspace{-0.7cm}
\subsection*{Intellectual Merit}
\vspace{-0.6cm}

The proposed research will leverage a series of ensemble high-resolution global model simulations to understand the impact of climate change on North Atlantic cyclones. This work will further focus on the entire lifecycle of cyclones, from genesis to extratropical transition and/or landfall, using the newly developed TempestExtremes framework for event detection and tracking.  The work will provide new understandings into i) the impact of African easterly waves on tropical cyclone formation in a warming world and ii) the influence of changes in storm characteristics due to climate change on extratropical transition and landfalling cyclones. Additionally, this project will support continued evaluation of high-resolution ($<$ 30 km) uniform-resolution and variable-resolution global models for the analysis of extreme weather events, such as tropical cyclones.  These next-generation models are a promising tool to understand the impact of anthropogenic climate change on extremes.  

\vspace{-0.7cm}
\subsection*{Broader Impacts}
\vspace{-0.6cm}

Landfalling tropical cyclones produce intense winds, heavy rain, high waves, and damaging storm surge in coastal locations. They are currently estimated to be responsible for 19,000 fatalities per year and \$26 billion/year in damages worldwide, making them one of the most devastating natural phenomena.  Understanding how cyclones will change in the coming decades is of significant value to science and society. It can be expected that damage due to cyclone landfalls in the future will increase significantly due to growth of coastal population and property.  As studies have reported a trend towards more intense tropical cyclones under a warmer climate, an understanding the changes in cyclone characteristics is crucial for disaster planning and adaptation in the United States -- particularly for cyclones in the North Atlantic basin.

\end{document}
